\documentclass{beamer}
\usetheme{CambridgeUS}

\title{Functions}
\author{Abhijeet}
\institute{IGIDR}
\date{\today}

\begin{document}

\begin{frame}[plain]
  \titlepage
\end{frame}

\begin{frame}{Relation}
\begin{itemize}
    \item When we talk about a \textbf{relation} $R$ from a \textbf{set} $A$ to a \textbf{set} $B$, we simply mean that $R$ is a subset of ordered pairs, where the first coordinate of the pair belongs to $A$, and the second coordinate belongs to $B$.
    \item Let's suppose $\textbf{A} = \{a, b, c\}$ and $\textbf{B} = \{1, 2, 3\}$. We now take a subset of the Cartesian product $\textbf{A} \times \textbf{B}$: $\{(a, 1), (a, 3), (c,2)\}$. We denote this subset by $R$.  
    \item $R = \{(a, 1), (a, 3), (c,2)\}$
\end{itemize}
\end{frame}


\begin{frame}{Terminologies}
    \begin{itemize}
        \item $dom(R)= \{a \in A : (a,b) \in R \quad\text{for some} \quad b \in B\}$
        \item $range(R)= \{b \in A : (a,b) \in R \quad\text{for some} \quad a \in A\}$
        \item $R^{-1}=\{(b,a):(b,a) \in  R\}$
        \item Q. Determine the inverse relation $R^{-1}$ for the relation $R = \{(x, y) : x + 4y \quad \text{is odd}\}$ defined on N.
        \item Let $A$ and $B$ be sets with $|A| = |B| = 4$.

        \begin{enumerate}
             \item Prove or disprove: If $R$ is a relation from $A$ to $B$ where $|R| = 9$ and $R = R^{-1}$, then $A = B$.
             \item Show that by making a small change in the statement in (a), a different response to the resulting statement can be obtained.
        \end{enumerate}
        \end{itemize} 

\end{frame}

\begin{frame}{Properties of Relations}
    \begin{itemize}
    \item A relation $R$ defined on a set $A$ is called \textbf{reflexive} if $x \ R \ x$ for every $x \in A$.
    \item A relation $R$ defined on a set $A$ is called \textbf{symmetric} if whenever $x \ R \ y$, then $y \ R \ x$ for all $x, y \in A$.
    \item A relation $R$ defined on a set $A$ is called \textbf{transitive} if whenever $x \ R \ y$ and $y \ R \ z$, then $x \ R \ z$, for all $x, y, z \in A$.
    \item A relation R defined on a set A is called an \textbf{equivalence} relation if R is reflexive, symmetric and transitive.
    \item Prove that The relation R defined on Z by x R y if x + 3y is even is an equivalence relation.
    \end{itemize}
\end{frame}

\begin{frame}{Functions}
    \begin{itemize}
    \item Let $A$ and $B$ be nonempty sets. By a \textbf{function} $f$ from $A$ to $B$, written $f: A \rightarrow B$, we mean a relation from $A$ to $B$ with the property that every element $a$ in $A$ is the first coordinate of exactly one ordered pair in $f$.
    \item Since $f$ is a relation, the set $A$ in this case is the \textbf{domain} of $f$, denoted by $\text{dom}(f)$. The set $B$ is called the \textbf{codomain} of $f$.
    \item Let $A$ and $B$ be nonempty sets. By a \textbf{correspondence} $R$ from $A$ to $B$, written $R: A \rightarrow B$, we mean a relation from $A$ to $B$ with the property that every element $a$ in $A$ is the first coordinate of at least one ordered pair in $R$.
    \item $\text{range}(f) = \{b \in B \mid \text{b is an image under } f \text{ of some element of } A\} = \{f(x) \mid x \in A\}$
    \end{itemize}
\end{frame}

\begin{frame}{One-to-One Functions(Injective)}
\begin{itemize}
    \item A function $f$ from a set $A$ to a set $B$ is called \textbf{one-to-one} or \textbf{injective} if every two distinct elements of $A$ have distinct images in $B$. In symbols, a function $f: A \to B$ is one-to-one if whenever $x, y \in A$ and $x \neq y$, then $f(x) \neq f(y)$.
    \item Thus, if a function $f: A \rightarrow B$ is not \textbf{one-to-one}, then there exist distinct elements $w$ and $z$ in $A$ such that $f(w) = f(z)$.
    \item Suppose that a function $f: A \rightarrow B$ is \textbf{one-to-one}, where $A$ and $B$ are finite sets. Since every two distinct elements of $A$ have distinct images in $B$, there must be at least as many elements in $B$ as in $A$, that is, $|A| \le |B|$.

    \item The function $f:R \to R$ defined by $y=x+2$ is a one to one function and $y=x^{2}$ is not.
\end{itemize}
\end{frame}

\begin{frame}{Onto Functions(Surjective)}
\begin{itemize}
    \item A function $f: A \rightarrow B$ is called \textbf{onto} or \textbf{surjective} if every element of the codomain $B$ is the image of some element of $A$. Equivalently, $f$ is onto if $f(A) = B$.
      \item For finite sets $A$ and $B$, a function $f: A \rightarrow B$ is \textbf{surjective} (or \textbf{onto}) if and only if $|B| \le |A|$.
    \item The function $f:R \to R$ defined by $y=x+2$ is a onto function.
    
\end{itemize}
\end{frame}

\begin{frame}{One-to-One and Onto Functions(Bijective)}
\begin{itemize}
    \item A function $f: A \rightarrow B$ is called \textbf{onto} or \textbf{surjective} if every element of the codomain $B$ is the image of some element of $A$. Equivalently, $f$ is onto if $f(A) = B$.
  
    \item The function $f:R \to R$ defined by $y=x+2$ is a onto function.
    
\end{itemize}
\end{frame}

\begin{frame}{Composition of Functions}
\begin{itemize}
    \item The composition $g \circ f$ of $f$ and $g$ is the function from $A$ to $C$ defined by $(g \circ f )(x) = g( f (x))$ for all $a \in A$.
    \item \textbf{Example:} Let $f(x) = \sin x$, and $g(x) = x^2$.
        $$ (f \circ g)(x) = f(g(x)) = f(x^2) = \sin(x^2) $$
        $$ (g \circ f)(x) = g(f(x)) = g(\sin x) = (\sin x)^2 = \sin^2 x $$
  
    \item Let $f : A \to B$ and $g : B \to C$ be two functions, then prove:
    \begin{enumerate}[(a)]
        \item If $f$ and $g$ are injective, then so is $g \circ f$.
         \item If $f$ and $g$ are surjective, then so is $g \circ f$.
    \end{enumerate}
    
\end{itemize}
\end{frame}


\begin{frame}{Inverse Functions}
\begin{itemize}
    \item For a relation $R$ from a set $A$ to a set $B$, the inverse relation $R^{-1}$ from $B$ to $A$ is defined as
$$R^{-1} = \{(b, a) : (a, b) \in R\}.$$
    \item \textbf{Example:} If $A = \{a, b, c, d\}$, $B = \{1, 2, 3\}$ and
$R = \{(a, 1), (a, 3), (c, 2), (c, 3), (d, 1)\}$\\
is a relation from $A$ to $B$, then
$R^{-1} = \{(1, a), (3, a), (2, c), (3, c), (1, d)\}.$

    \item Let $f : A \to B$ be a function. Then the inverse relation $f^{-1}$ is a function from $B$ to $A$ if and only if $f$ is bijective. Furthermore, if $f$ is bijective, then $f^{-1}$ is also bijective.
   
    \item For all values in its Domain, $f(x)=x^{2}$ doesn't have an inverse 
\end{itemize}
\end{frame}

\end{document}