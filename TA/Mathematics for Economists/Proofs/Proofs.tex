\documentclass{beamer}
\usetheme{CambridgeUS}
\usecolortheme{beaver} % Red-based color theme
\setbeamertemplate{headline}{}
% Packages
\usepackage{amsmath,amssymb}
\usepackage{graphicx}
\usepackage{booktabs} % For professional tables
\usepackage{array}
\usepackage{graphicx}

% Title page info
\title{Mathematics for Economists}
\subtitle{Proof Techniques }
\author{Abhijeet}
\institute{IGIDR}
\date{\today}

% ================================================
\begin{document}

% Title frame
\begin{frame}
    \titlepage
\end{frame}

% Table of contents
\begin{frame}{Outline}
    \tableofcontents
\end{frame}

% Section: Logic
\section{Some Definitions}

\begin{frame}{Some Definition}
\begin{itemize}
    \item An \textbf{axiom} is a true mathematical statement whose truth is
    accepted without proof.
    \item A \textbf{theorem} is a true mathematical statement whose truth can be
    verified.
    \item A \textbf{corollary} is a mathematical result that can be deduced
    from, and is thereby a consequence of, some earlier result.
    \item A \textbf{lemma} is a mathematical result that is useful in
    establishing the truth of some other result.
\end{itemize}
\end{frame}

% Section: Calculus
\section{Direct Proof}

\begin{frame}{Direct Proof}
\begin{itemize}
    \item A \textbf{direct proof} is a method of proving a mathematical
    statement by a straightforward chain of logical deductions from known facts,
    definitions, and previously established results, without assuming the
    contrary of what is to be proved.
    \item A \textbf{direct proof} is a structured way of proving a mathematical
    statement of the form:
\[
P \implies Q
\]
where:
\begin{itemize}
    \item \( P \) is the \textbf{hypothesis} (assumption), and
    \item \( Q \) is the \textbf{conclusion} (what we want to prove).
\end{itemize}
We assume $P(x)$ to be true and apply reasoning, definitions, and other results
to conclude that $Q(x)$ is true.
\end{itemize}
\end{frame}




\begin{frame}{Direct Proof: Example}
\textbf{(1) Using Direct Proof method, prove that, if $n$ is even integer then
$n^{2}$ is also even integer.}\\ \textbf{Proof:} We need to prove the claim is
true given the assumption is true. What do we know about the assumption? We know
that $n$ is even. So we can write $n=2k, k \in Z$. Now square it. We get,
$n^{2}=4k^{2}=2 \times 2k^{2}$. Now $2k^{2}$ is an integer and that multiplied
by 2 yield an even integer. QED\\ 

\textbf{(2) Using Direct Proof Method, prove that, if $n$ is an odd integer,
then $3n+7$ is an even integer.} \\ 
\textbf{Proof:} We know $n$ is an odd integer. So we can write, $n=2k+1$, where
$k \in Z$. Substituting the value of $n=2k+1$ in $3n+7$, we get, $6k+10=(2
\times 3k+5)$, $3k+5$ is an ineteger, that multiplied by $2$ yield an even
integer.  $6k+8=3n+7$, given $n=2k+1, k \in Z$ is even. QED 

\end{frame}


\section{Proof by Contrapositive}

\begin{frame}{Proof by Contrapositive}
\begin{itemize}
    \item For statements \( P \) and \( Q \), the contrapositive of the
    implication \( P \implies Q \) is the implication
$
\neg Q \implies \neg P
$
    \item For statements \( P \) and \( Q \), the contrapositive of the
    implication \( P \implies Q \) is the implication
$
\neg Q \implies \neg P
$
\item If $Q$  is false, then $ P$ is also false.
\end{itemize}
\end{frame}

\begin{frame}{Proof by Contrapositive: Examples}
\textbf{(1) Using proof by contrapositive, Prove that if $5x-7$ is even then $x$
is odd.}\\ 

\textbf{Proof:} We assume, $x$ to be not odd, that is $x$ is even, and we will
try to prove that if $x$ is even then, $5x-7$ is odd. We can write, $x=2k, k \in
Z$. So, $5x-7=10k-7=(2 \times 5k)+7$. Now even+odd will produce us odd. Thus,
$(10k-7)=5x-7$ is odd. QED \\

\textbf{(2)Let \( x \in \mathbb{Z} \). Then \( x^2 \) is even if and only if \(
x \) is even:
$
x^2 \text{ is even} \iff x \text{ is even}
$}

\textbf{Proof:} In biconditional statements, we need to prove both ways, that is
$A \implies B$ and($\wedge$) $B \implies A$\\ 
So, (i) If $x^2$ is even then $x$ is even and (ii) If $x$ is even, then $x^2$ is
even.\\ 

\textbf{(3) Let $x \in Z$. If $5x-7$ is odd, then $9x+2$ is even.}
\vspace{0.2cm}
Sometimes, its not possible to prove a result by assuming values for all $x$.
So, we divide into cases, like, $\text{Case 1:}$ If $x$ is even then $B$,
$\textbf{Case 2:}$ If $x$ is odd then $B$.
\end{frame}


\section{Existence and Proof by Contradiction}

\begin{frame}{Counter-examples}

If the statement $\forall x \in S, R(x)$ is false, then there exists some
element $x \in S$ for which $R(x)$ is false:  
$\neg (\forall x \in S, R(x)) \equiv \exists x \in S \text{ such that } \neg
R(x)$.  

Such an element $x$ is called a \textbf{counter-example} of the (false)
statement $\forall x \in S, R(x)$.  

Finding a counter-example verifies that $\forall x \in S, R(x)$ is false.

\end{frame}

\begin{frame}{Counter-examples: Examples}
\textbf{Disprove the statement:}  
Let $n \in \mathbb{Z}$.  If $n^2 + 3n$ is even, then $n$ is odd.\\ 

Take $n=2$. Now, $n^2 + 3n=10$, which is even but $n$ is odd. Hence, we disproved.

\end{frame}

\section{Proof by Contradiction}
\begin{frame}{Proof by Contradiction}
To prove a statement $P$, we assume the negation $\neg P$ and show that this leads to a contradiction.  
That is, we show:
\[
\neg P \implies \text{False}
\]
Therefore, $P$ must be true.\\ 

If $R$ is the quantified statement  
\[
R: \forall x \in S, \; P(x) \implies Q(x),
\]
then a proof by contradiction might begin with the assumption:  

\textit{Assume, to the contrary, that there exists some element $x \in S$ for which $P(x)$ is true and $Q(x)$ is false}, i.e.,  
\[
\exists x \in S \text{ such that } P(x) \text{ is true and } Q(x) \text{ is false}.
\]

The remainder of the proof then consists of showing that this assumption leads to a contradiction.

\end{frame}

\begin{frame}{Proof by Contradiction: Examples}
\textbf{(1) By using the method of 'Proof by contradiction', prove that: There is no smallest positive real number.}\\
\textbf{(2) No odd integer can be expressed as the sum of three even integer.}\\
\textbf{(3) Prove that if $x$ and $y$ are positive real numbers, then $
\sqrt{x} + \sqrt{y} \neq \sqrt{2(x + y)}$}\\ 
\textbf{(4) Prove that there do not exist three distinct real numbers $a$, $b$, and $c$ such that all of the numbers  $
a + b + c, \quad ab, \quad ac, \quad bc, \quad abc$
are equal.}
\end{frame}


\section{How to prove (and How not to): $\forall \in S, P(x) \implies Q(x)$}
\begin{frame}{How to prove (and How not to): $\forall \in S, P(x) \implies Q(x)$}


\scriptsize % reduces font size

\resizebox{\textwidth}{!}{%
\begin{tabular}{|c|p{6cm}|p{6cm}|}
\hline
\textbf{No.} & \textbf{First Step of “Proof”} & \textbf{Remarks/Goal} \\
\hline
1 & Assume for an arbitrary element $x \in S$ that $P(x)$ is true. & A direct proof is being used. Show that $Q(x)$ is true for the element $x$. \\
\hline
2 & Assume for an arbitrary element $x \in S$ that $P(x)$ is false. & A mistake has been made. \\
\hline
3 & Assume for an arbitrary element $x \in S$ that $Q(x)$ is true. & A mistake has been made. \\
\hline
4 & Assume for an arbitrary element $x \in S$ that $Q(x)$ is false. & A proof by contrapositive is being used. Show that $P(x)$ is false for the element $x$. \\
\hline
5 & Assume for an arbitrary element $x \in S$ that $P(x)$ and $Q(x)$ are true. & A mistake has been made. \\
\hline
6 & Assume that there exists $x \in S$ such that $P(x)$ is true and $Q(x)$ is false. & A proof by contradiction is being used. Produce a contradiction. \\
\hline
7 & Assume that there exists $x \in S$ such that $P(x)$ is false and $Q(x)$ is true. & A mistake has been made. \\
\hline
8 & Assume that there exists $x \in S$ such that $P(x)$ and $Q(x)$ are false. & A proof by contradiction is being used. Produce a contradiction. \\
\hline
9 & Assume that there exists $x \in S$ such that $P(x) \implies Q(x)$ is true. & A mistake has been made. \\
\hline
10 & Assume that there exists $x \in S$ such that $P(x) \implies Q(x)$ is false. & A mistake has been made. \\
\hline
\end{tabular}%
}

\end{frame}
\section{Proof by Mathematical Induction}

\begin{frame}{Proof by Mathematical Induction}
For each positive integer $n$, let $P(n)$ be a statement.  
If 

\begin{enumerate}
    \item $P(1)$ is true, and
    \item $\forall k \in \mathbb{N}, \; P(k) \implies P(k+1)$ is true,
\end{enumerate}

then $\forall n \in \mathbb{N}, \; P(n)$ is true.

As a consequence of the Principle of Mathematical Induction, the quantified statement $\forall n \in \mathbb{N}, \; P(n)$ can be proved to be true if

\begin{enumerate}
    \item we can show that the statement $P(1)$ is true, and
    \item we can establish the truth of the implication
    \[
    P(k) \implies P(k+1)
    \]
    for every positive integer $k$.
\end{enumerate}

\end{frame}

\begin{frame}{Proof by Mathematical Induction: Examples}
\textbf{Prove the following using Mathematical Induction:}
\begin{itemize}
    \item Find a formula for $1 + 4 + 7 + \cdots + (3n-2)$ for positive integers $n$, and verify your formula by mathematical induction.
    \item Prove that $1 \cdot 3 + 2 \cdot 4 + 3 \cdot 5 + \cdots + n(n+2) = \frac{n(n+1)(2n+7)}{6}$ for every positive integer $n$.
    \item Prove that $1 + 2 + 3 + \cdots + n = \frac{n(n+1)}{2}$ for every positive integer $n$.
    \item Prove that $1^2 + 2^2 + \cdots + n^2 = \frac{n(n+1)(2n+1)}{6}$ for every positive integer $n$.
\end{itemize}

\end{frame}
\end{document}