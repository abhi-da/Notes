\documentclass{beamer}

\usepackage{graphicx}
\usepackage{amsmath}
\usepackage{amssymb}

\usetheme{Madrid}
\usecolortheme{default}

\title{Boolean Algebra}
\author{Your Name}
\date{\today}

\begin{document}

\begin{frame}
    \titlepage
\end{frame}

\begin{frame}{Table of Contents}
    \tableofcontents
\end{frame}

\section{Introduction}
\begin{frame}{What is Boolean Algebra?}
    \begin{itemize}
        \item A branch of algebra in which the values of the variables are the
        truth values \textit{true} and \textit{false}, usually denoted 1 and 0
        respectively.
        \item It is used to analyze and simplify digital circuits.
        \item It is also called Binary Algebra or logical Algebra.
        \item George Boole, an English mathematician, developed this algebra in
        1854.
    \end{itemize}
\end{frame}

\section{Basic Operations}
\begin{frame}{Basic Operations}
    The basic operations in Boolean algebra are:
    \begin{itemize}
        \item \textbf{Conjunction} ($\land$): Corresponds to the AND operator.
        \item \textbf{Disjunction} ($\lor$): Corresponds to the OR operator.
        \item \textbf{Negation} ($\neg$): Corresponds to the NOT operator.
    \end{itemize}
\end{frame}

\begin{frame}{AND Operator}
    The AND operator is denoted by a dot ($\cdot$) or by the absence of an
    operator.
    \begin{itemize}
        \item $A \cdot B = Y$ or $AB = Y$
        \item The expression is true (1) only if both A and B are true (1).
    \end{itemize}
    \begin{table}
        \centering
        \begin{tabular}{|c|c|c|}
            \hline
            \textbf{A} & \textbf{B} & \textbf{A AND B} \\
            \hline
            0 & 0 & 0 \\
            0 & 1 & 0 \\
            1 & 0 & 0 \\
            1 & 1 & 1 \\
            \hline
        \end{tabular}
        \caption{Truth Table for AND}
    \end{table}
\end{frame}

\begin{frame}{OR Operator}
    The OR operator is denoted by a plus sign (+).
    \begin{itemize}
        \item $A + B = Y$
        \item The expression is true (1) if either A or B or both are true (1).
    \end{itemize}
    \begin{table}
        \centering
        \begin{tabular}{|c|c|c|}
            \hline
            \textbf{A} & \textbf{B} & \textbf{A OR B} \\
            \hline
            0 & 0 & 0 \\
            0 & 1 & 1 \\
            1 & 0 & 1 \\
            1 & 1 & 1 \\
            \hline
        \end{tabular}
        \caption{Truth Table for OR}
    \end{table}
\end{frame}

\begin{frame}{NOT Operator}
    The NOT operator is denoted by a prime ($A'$).
    \begin{itemize}
        \item $A' = Y$
        \item It inverts the input. If the input is true (1), the output is
        false (0).
    \end{itemize}
    \begin{table}
        \centering
        \begin{tabular}{|c|c|}
            \hline
            \textbf{A} & \textbf{NOT A} \\
            \hline
            0 & 1 \\
            1 & 0 \\
            \hline
        \end{tabular}
        \caption{Truth Table for NOT}
    \end{table}
\end{frame}

\section{Laws of Boolean Algebra}
\begin{frame}{Basic Laws}
    \begin{columns}
        \begin{column}{0.5\textwidth}
            \textbf{Commutative Law}
            \begin{itemize}
                \item $A + B = B + A$
                \item $A \cdot B = B \cdot A$
            \end{itemize}
            \textbf{Associative Law}
            \begin{itemize}
                \item $(A + B) + C = A + (B + C)$
                \item $(A \cdot B) \cdot C = A \cdot (B \cdot C)$
            \end{itemize}
        \end{column}
        \begin{column}{0.5\textwidth}
            \textbf{Distributive Law}
            \begin{itemize}
                \item $A \cdot (B + C) = A \cdot B + A \cdot C$
                \item $A + (B \cdot C) = (A + B) \cdot (A + C)$
            \end{itemize}
        \end{column}
    \end{columns}
\end{frame}

\begin{frame}{Other Important Laws}
    \begin{itemize}
        \item \textbf{Identity Law}: $A + 0 = A$, $A \cdot 1 = A$
        \item \textbf{Annulment Law}: $A + 1 = 1$, $A \cdot 0 = 0$
        \item \textbf{Idempotent Law}: $A + A = A$, $A \cdot A = A$
        \item \textbf{Complement Law}: $A + A' = 1$, $A \cdot A' = 0$
        \item \textbf{Involution Law}: $(A')' = A$
        \item \textbf{De Morgan's Laws}:
              \begin{itemize}
                \item $(A + B)' = A' \cdot B'$
                \item $(A \cdot B)' = A' + B'$
              \end{itemize}
    \end{itemize}
\end{frame}


\section{Minterms and Maxterms}

\begin{frame}{Minterms}
    \begin{itemize}
        \item A minterm is a product term (AND operation) that contains all
        variables of the function, either in their normal or complemented form.
        \item For a given row of a truth table, the minterm is 1 if and only if
        the input combination is the one for that row.
        \item A variable appears in its normal form ($X$) if its value is 1 in
        the input combination.
        \item A variable appears in its complemented form ($X'$) if its value is
        0 in the input combination.
        \item Example for two variables A and B:
            \begin{itemize}
                \item $A'B'$ (for A=0, B=0)
                \item $A'B$ (for A=0, B=1)
                \item $AB'$ (for A=1, B=0)
                \item $AB$ (for A=1, B=1)
            \end{itemize}
    \end{itemize}
\end{frame}

\begin{frame}{Maxterms}
    \begin{itemize}
        \item A maxterm is a sum term (OR operation) that contains all variables
        of the function, either in their normal or complemented form.
        \item For a given row of a truth table, the maxterm is 0 if and only if
        the input combination is the one for that row.
        \item A variable appears in its normal form ($X$) if its value is 0 in
        the input combination.
        \item A variable appears in its complemented form ($X'$) if its value is
        1 in the input combination.
        \item Example for two variables A and B:
            \begin{itemize}
                \item $A+B$ (for A=0, B=0)
                \item $A+B'$ (for A=0, B=1)
                \item $A'+B$ (for A=1, B=0)
                \item $A'+B'$ (for A=1, B=1)
            \end{itemize}
    \end{itemize}
\end{frame}

\begin{frame}{Sum of Products (SOP)}
    \begin{itemize}
        \item A Boolean expression can be represented as a sum of minterms.
        \item This form is called the Sum of Products (SOP) or Disjunctive
        Normal Form (DNF).
        \item To get the SOP form from a truth table, we sum (OR) all the
        minterms for which the function's output is 1.
    \end{itemize}
\end{frame}

\begin{frame}{Product of Sums (POS)}
    \begin{itemize}
        \item A Boolean expression can also be represented as a product of
        maxterms.
        \item This form is called the Product of Sums (POS) or Conjunctive
        Normal Form (CNF).
        \item To get the POS form from a truth table, we multiply (AND) all the
        maxterms for which the function's output is 0.
    \end{itemize}
\end{frame}

\begin{frame}{Example: Generating Compound Statements}
    Let's consider the compound statement "If A, then B", which is denoted as $A
    \rightarrow B$. The truth table is as follows:
    \begin{table}
        \centering
        \begin{tabular}{|c|c|c|}
            \hline
            \textbf{A} & \textbf{B} & \textbf{A $\rightarrow$ B} \\
            \hline
            0 & 0 & 1 \\
            0 & 1 & 1 \\
            1 & 0 & 0 \\
            1 & 1 & 1 \\
            \hline
        \end{tabular}
        \caption{Truth Table for $A \rightarrow B$}
    \end{table}
\end{frame}

\begin{frame}{Example: SOP and POS from Truth Table}
    From the truth table for $A \rightarrow B$:
    \begin{itemize}
        \item \textbf{Sum of Products (SOP):} We look for rows where the output
        is 1.
        \begin{itemize}
            \item Row 1 (A=0, B=0): Minterm is $A'B'$
            \item Row 2 (A=0, B=1): Minterm is $A'B$
            \item Row 4 (A=1, B=1): Minterm is $AB$
        \end{itemize}
        The SOP expression is $Y = A'B' + A'B + AB$. This can be simplified to
        $A' + B$.
        
        \item \textbf{Product of Sums (POS):} We look for rows where the output
        is 0.
        \begin{itemize}
            \item Row 3 (A=1, B=0): Maxterm is $A'+B$
        \end{itemize}
        The POS expression is $Y = A'+B$.
    \end{itemize}
    In this case, the simplified SOP and the POS are the same, which is often
    the case. The expression for "If A, then B" is logically equivalent to "Not
    A or B".
\end{frame}

\section{Simplification of Boolean Expressions}
\begin{frame}{Derivation of the Simplification}
    Let's derive the simplification of the expression $Y = A'B' + A'B + AB$.
    \begin{align*}
        Y &= A'B' + A'B + AB \\
          &= A'(B' + B) + AB && \text{Distributive Law} \\
          &= A'(1) + AB && \text{Complement Law: $B + B' = 1$} \\
          &= A' + AB && \text{Identity Law: $A \cdot 1 = A$} \\
          &= (A' + A)(A' + B) && \text{Distributive Law} \\
          &= (1)(A' + B) && \text{Complement Law: $A + A' = 1$} \\
          &= A' + B && \text{Identity Law: $1 \cdot A = A$}
    \end{align*}
    This shows that the expression $A'B' + A'B + AB$ simplifies to $A' + B$.
    This is a common simplification that is useful to remember, and it
    corresponds to the logical statement for implication, $A \rightarrow B
    \equiv \neg A \lor B$.
\end{frame}

\begin{frame}{For statements $P$ and $Q$, show that $P$ implies $(P \vee Q)$ is a tautology (1)}
We can write the above expression as: \[
P' + (P + Q)
\]

We know that 
\[
A \Rightarrow B \equiv A' + B
\] 
(that is, not $A$ or $B$).  

Hence,
\[
P \Rightarrow (P + Q) \equiv P' + (P + Q)
\]

Using the associative law:
\[
P' + (P + Q) = (P' + P) + Q
\]
\end{frame}

\begin{frame}{For statements $P$ and $Q$, show that $P$ implies $(P \vee Q)$ is a tautology (2)}
By the annulment law:
\[
(P' + P) + Q = 1 + Q
\]

Again, by the annulment law:
\[
1 + Q = 1
\]

\alert{Thus, $P \Rightarrow (P + Q)$ is always true, so it is a tautology.}
\end{frame}

\begin{frame}{For statements $P$ and $Q$, show that $(P \cdot Q') \cdot (P \cdot Q)$ is a contradiction}

Start with the Boolean expression:  
\[
(P \cdot Q') \cdot (P \cdot Q)
\]

Rearrange terms:  
\[
(P \cdot Q') \cdot (Q \cdot P) \quad \text{(Commutative Law)}
\]

Group terms:  
\[
P \cdot (Q' \cdot Q) \cdot P \quad \text{(Associative Law)}
\]

Simplify complements:  
\[
P \cdot 0 \cdot P = 0 \quad \text{($Q' \cdot Q = 0$, Complement Law)}
\]


\alert{Hence, $(P \cdot Q') \cdot (P \cdot Q)$ is always false, so it is a contradiction.}
\end{frame}

\begin{frame}{Show that $(P \Rightarrow \neg Q) \cdot (P \cdot Q)$ is a contradiction}

Convert the implication using the formula $A \Rightarrow B \equiv A' + B$:  
\[
P \Rightarrow \neg Q \equiv P' + Q'
\]

So the statement becomes:  
\[
(P' + Q') \cdot (P \cdot Q)
\]

Apply distributive law:  
\[
(P' + Q') \cdot (P \cdot Q) = P \cdot Q \cdot P' + P \cdot Q \cdot Q' \quad \text{(Distributive Law)}
\]

Simplify using complement law:  
\[
P \cdot Q \cdot P' + P \cdot Q \cdot Q' = 0 + 0 = 0 \quad \text{(Complement Law, $P \cdot P' = 0$, $Q \cdot Q' = 0$)}
\]

\alert{Hence, $(P \Rightarrow \neg Q) \cdot (P \cdot Q)$ is always false, so it is a contradiction.}
\end{frame}



\end{document}
