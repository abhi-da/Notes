\documentclass{beamer}

\usepackage{graphicx}
\usepackage{amsmath}
\usepackage{amssymb}

\usetheme{Madrid}
\usecolortheme{default}

\title{Boolean Algebra}
\author{Your Name}
\date{\today}

\begin{document}

\begin{frame}
    \titlepage
\end{frame}

\begin{frame}{Table of Contents}
    \tableofcontents
\end{frame}

\section{Introduction}
\begin{frame}{What is Boolean Algebra?}
    \begin{itemize}
        \item A branch of algebra in which the values of the variables are the truth values \textit{true} and \textit{false}, usually denoted 1 and 0 respectively.
        \item It is used to analyze and simplify digital circuits.
        \item It is also called Binary Algebra or logical Algebra.
        \item George Boole, an English mathematician, developed this algebra in 1854.
    \end{itemize}
\end{frame}

\section{Basic Operations}
\begin{frame}{Basic Operations}
    The basic operations in Boolean algebra are:
    \begin{itemize}
        \item \textbf{Conjunction} ($\land$): Corresponds to the AND operator.
        \item \textbf{Disjunction} ($\lor$): Corresponds to the OR operator.
        \item \textbf{Negation} ($\neg$): Corresponds to the NOT operator.
    \end{itemize}
\end{frame}

\begin{frame}{AND Operator}
    The AND operator is denoted by a dot ($\cdot$) or by the absence of an operator.
    \begin{itemize}
        \item $A \cdot B = Y$ or $AB = Y$
        \item The expression is true (1) only if both A and B are true (1).
    \end{itemize}
    \begin{table}
        \centering
        \begin{tabular}{|c|c|c|}
            \hline
            \textbf{A} & \textbf{B} & \textbf{A AND B} \\
            \hline
            0 & 0 & 0 \\
            0 & 1 & 0 \\
            1 & 0 & 0 \\
            1 & 1 & 1 \\
            \hline
        \end{tabular}
        \caption{Truth Table for AND}
    \end{table}
\end{frame}

\begin{frame}{OR Operator}
    The OR operator is denoted by a plus sign (+).
    \begin{itemize}
        \item $A + B = Y$
        \item The expression is true (1) if either A or B or both are true (1).
    \end{itemize}
    \begin{table}
        \centering
        \begin{tabular}{|c|c|c|}
            \hline
            \textbf{A} & \textbf{B} & \textbf{A OR B} \\
            \hline
            0 & 0 & 0 \\
            0 & 1 & 1 \\
            1 & 0 & 1 \\
            1 & 1 & 1 \\
            \hline
        \end{tabular}
        \caption{Truth Table for OR}
    \end{table}
\end{frame}

\begin{frame}{NOT Operator}
    The NOT operator is denoted by a prime ($A'$).
    \begin{itemize}
        \item $A' = Y$
        \item It inverts the input. If the input is true (1), the output is false (0).
    \end{itemize}
    \begin{table}
        \centering
        \begin{tabular}{|c|c|}
            \hline
            \textbf{A} & \textbf{NOT A} \\
            \hline
            0 & 1 \\
            1 & 0 \\
            \hline
        \end{tabular}
        \caption{Truth Table for NOT}
    \end{table}
\end{frame}

\section{Laws of Boolean Algebra}
\begin{frame}{Basic Laws}
    \begin{columns}
        \begin{column}{0.5\textwidth}
            \textbf{Commutative Law}
            \begin{itemize}
                \item $A + B = B + A$
                \item $A \cdot B = B \cdot A$
            \end{itemize}
            \textbf{Associative Law}
            \begin{itemize}
                \item $(A + B) + C = A + (B + C)$
                \item $(A \cdot B) \cdot C = A \cdot (B \cdot C)$
            \end{itemize}
        \end{column}
        \begin{column}{0.5\textwidth}
            \textbf{Distributive Law}
            \begin{itemize}
                \item $A \cdot (B + C) = A \cdot B + A \cdot C$
                \item $A + (B \cdot C) = (A + B) \cdot (A + C)$
            \end{itemize}
        \end{column}
    \end{columns}
\end{frame}

\begin{frame}{Other Important Laws}
    \begin{itemize}
        \item \textbf{Identity Law}: $A + 0 = A$, $A \cdot 1 = A$
        \item \textbf{Annulment Law}: $A + 1 = 1$, $A \cdot 0 = 0$
        \item \textbf{Idempotent Law}: $A + A = A$, $A \cdot A = A$
        \item \textbf{Complement Law}: $A + A' = 1$, $A \cdot A' = 0$
        \item \textbf{Involution Law}: $(A')' = A$
        \item \textbf{De Morgan's Laws}:
              \begin{itemize}
                \item $(A + B)' = A' \cdot B'$
                \item $(A \cdot B)' = A' + B'$
              \end{itemize}
    \end{itemize}
\end{frame}

\section{Logic Gates}
\begin{frame}{Logic Gates}
    \begin{itemize}
        \item Logic gates are the basic building blocks of any digital system.
        \item They are electronic circuits having one or more than one input and only one output.
        \item The relationship between the input and the output is based on a certain logic.
        \item Basic logic gates are: AND, OR, NOT gates.
    \end{itemize}
\end{frame}

\begin{frame}{Conclusion}
    \begin{itemize}
        \item Boolean algebra is fundamental to the design of digital circuits.
        \item It provides a way to simplify complex logic expressions.
        \item The basic operations are AND, OR, and NOT, which are represented by corresponding logic gates.
    \end{itemize}
\end{frame}

\end{document}
