\documentclass{beamer}
\usetheme{CambridgeUS}
\usecolortheme{beaver} % Red-based color theme

% Packages
\usepackage{amsmath,amssymb}
\usepackage{graphicx}
\usepackage{booktabs} % For professional tables

% Title page info
\title{Mathematics for Economists}
\subtitle{Proof Techniques }
\author{Abhijeet}
\institute{IGIDR}
\date{\today}

% ================================================
\begin{document}

% Title frame
\begin{frame}
    \titlepage
\end{frame}

% Table of contents
\begin{frame}{Outline}
    \tableofcontents
\end{frame}

% Section: Logic
\section{Logic and Proof Techniques}

\begin{frame}{Logical Implication}
    \begin{theorem}[Transitivity of Implication]
        For statements $P$, $Q$, and $R$, the following is a tautology:
        \[
        ((P \Rightarrow Q) \land (Q \Rightarrow R)) \Rightarrow (P \Rightarrow R)
        \]
    \end{theorem}

    \begin{proof}
        Using Boolean algebra:
        \begin{align*}
            &((P \Rightarrow Q) \land (Q \Rightarrow R)) \Rightarrow (P \Rightarrow R) \\
            &\equiv ((\neg P \lor Q) \land (\neg Q \lor R)) \Rightarrow (\neg P \lor R) \\
            &\equiv \neg[(\neg P \lor Q) \land (\neg Q \lor R)] \lor (\neg P \lor R) \\
            &\equiv (P \land \neg Q) \lor (Q \land \neg R) \lor \neg P \lor R \\
            &\equiv \neg P \lor R \lor (P \land \neg Q) \lor (Q \land \neg R) \\
            &\equiv (\neg P \lor (P \land \neg Q)) \lor (R \lor (Q \land \neg R)) \\
            &\equiv (\neg P \lor \neg Q) \lor (R \lor Q) \\
            &\equiv \neg P \lor (\neg Q \lor Q) \lor R \\
            &\equiv \neg P \lor 1 \lor R \equiv 1.
        \end{align*}
    \end{proof}
\end{frame}

% Section: Calculus
\section{Calculus}

\begin{frame}{Mean Value Theorem}
    \begin{theorem}[Mean Value Theorem]
        Let $f$ be continuous on $[a,b]$ and differentiable on $(a,b)$. Then there exists $c \in (a,b)$ such that
        \[
        f'(c) = \frac{f(b) - f(a)}{b - a}.
        \]
    \end{theorem}
    % Remove the image if you don't have it
\end{frame}

% Section: Linear Algebra
\section{Linear Algebra}

\begin{frame}{Eigenvalues and Eigenvectors}
    \begin{definition}[Eigenvalue and Eigenvector]
        Let $A$ be an $n \times n$ matrix. A scalar $\lambda$ is an eigenvalue of $A$ if there exists a nonzero vector $\mathbf{v}$ such that
        \[
        A\mathbf{v} = \lambda \mathbf{v}.
        \]
    \end{definition}

    \begin{example}
        Let $A = \begin{bmatrix} 2 & 1 \\ 1 & 2 \end{bmatrix}$. Then $\lambda = 3$ is an eigenvalue with eigenvector $\mathbf{v} = \begin{bmatrix} 1 \\ 1 \end{bmatrix}$.
    \end{example}
\end{frame}

% Conclusion
\begin{frame}{Summary}
    \begin{itemize}
        \item Covered key proofs in logic, calculus, and linear algebra.
        \item Used Boolean algebra for logical tautologies.
        \item Presented fundamental theorems with examples.
    \end{itemize}
\end{frame}

\end{document}