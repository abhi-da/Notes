\documentclass{beamer}

% Theme
\usetheme{CambridgeUS}

% Packages
\usepackage[utf8]{inputenc}
\usepackage{amsmath}
\usepackage{amssymb}
\usepackage{amsthm}

% Theorem environments
\theoremstyle{plain}
\newtheorem{theorem}{Theorem}
\newtheorem{corollary}[theorem]{Corollary}
\newtheorem{lemma}[theorem]{Lemma}
\newtheorem{proposition}[theorem]{Proposition}

\theoremstyle{definition}
\newtheorem{definition}{Definition}
\newtheorem{example}{Example}

\theoremstyle{remark}
\newtheorem{remark}{Remark}

% Title Information
\title{Presentation on Mathematical Proofs}
\author{Your Name}
\institute{Your Institution}
\date{\today}

\begin{document}

% Title Frame
\begin{frame}
    \titlepage
\end{frame}

% Table of Contents
\begin{frame}{Outline}
    \tableofcontents
\end{frame}

\section{Introduction}

\begin{frame}{What is a Proof?}
    \begin{definition}
        A mathematical proof is a deductive argument for a mathematical statement, showing that the stated assumptions logically guarantee the conclusion.
    \end{definition}
    \begin{itemize}
        \item It must be based on axioms and previously established statements.
        \item The argument must be universally valid.
    \end{itemize}
\end{frame}

\section{Example Proof}

\begin{frame}{Theorem and Proof}
    \begin{theorem}[Pythagorean Theorem]
        For a right-angled triangle with legs of length $a$ and $b$ and hypotenuse of length $c$, the following relationship holds:
        \[ a^2 + b^2 = c^2 \]
    \end{theorem}

    \begin{proof}
        Consider a square with side length $a+b$. We can arrange four copies of the right-angled triangle in the corners of this square.
        \begin{itemize}
            \item The area of the large square is $(a+b)^2 = a^2 + 2ab + b^2$.
            \item The area of the four triangles is $4 \times \frac{1}{2}ab = 2ab$.
            \item The area of the inner square (formed by the hypotenuses) is $c^2$.
        \end{itemize}
        The area of the large square is also the sum of the areas of the four triangles and the inner square.
        \begin{align*}
            (a+b)^2 &= 4\left(\frac{1}{2}ab\right) + c^2 \\
            a^2 + 2ab + b^2 &= 2ab + c^2 \\
            a^2 + b^2 &= c^2
        \end{align*}
        This concludes the proof.
    \end{proof}
\end{frame}

\section{Conclusion}

\begin{frame}{Summary}
    \begin{itemize}
        \item We defined what a mathematical proof is.
        \item We saw an example of a proof for the Pythagorean Theorem.
        \item Beamer provides a great environment for presenting mathematical content.
    \end{itemize}
\end{frame}

\end{document}
