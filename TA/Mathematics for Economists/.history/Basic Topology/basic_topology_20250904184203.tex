\documentclass{beamer}
\usepackage{amsmath}

% Theme
\usetheme{Madrid}
\usecolortheme{beaver}

\setbeamertemplate{theorems}[numbered]

% Title info
\title[Real Analysis]{Real Analysis}
\subtitle{Basic Topology}
\author[Abhijeet]{Abhijeet \\ \small abhijeet.m@igidr.ac.in}
\institute[IGIDR]{Mathematics for Economists \\  IGIDR}
\date{\today}

\begin{document}

% Title slide
\begin{frame}[plain]
  \titlepage
\end{frame}

% A simple second slide
\begin{frame}{Metric Spaces}
  \begin{itemize}

    \item A metric space is a set $X$ together with a function $d: X \times X
    \to \mathbb{R}$ satisfying:
    \begin{itemize}
      \item Non-negativity: $d(x,y) \geq 0$ and $d(x,y) = 0$ iff $x = y$.
      \item Symmetry: $d(x,y) = d(y,x)$.
      \item Triangle inequality: $d(x,z) \leq d(x,y) + d(y,z)$.
    \end{itemize}
    \item Example: The Euclidean space $\mathbb{R}^n$ with the standard distance
    $d(x,y) = \sqrt{\sum_{i=1}^n (x_i - y_i)^2}$.
    \item Metric spaces provide a framework for discussing concepts like
    convergence, continuity, and compactness.
  \end{itemize}
\end{frame}

\begin{frame}{Metric Spaces}
 Let $X$ be a metric space. All points and sets below are understood to be in
 $X$.


\begin{enumerate}[(a)]
    \item A \textbf{neighborhood} of $p$ is a set $N_r(p)$ consisting of all $q$
    such that $d(p,q) < r$ for some $r > 0$. The number $r$ is called the
    \emph{radius} of $N_r(p)$.

    \item A point $p$ is a \textbf{limit point} of the set $E$ if every
    neighborhood of $p$ contains a point $q \neq p$ such that $q \in E$.

    \item If $p \in E$ and $p$ is not a limit point of $E$, then $p$ is called
    an \textbf{isolated point} of $E$.

    \item $E$ is \textbf{closed} if every limit point of $E$ belongs to $E$.

    \item A point $p$ is an \textbf{interior point} of $E$ if there exists a
    neighborhood $N$ of $p$ such that $N \subseteq E$.

    \item $E$ is \textbf{open} if every point of $E$ is an interior point.
\end{enumerate}
\end{frame}
\begin{frame}{Metric Spaces}
\begin{enumerate}[(a)]
    \item The \textbf{complement} of $E$ (denoted $E^c$) is the set of all
    points $p \in X$ such that $p \notin E$.

    \item $E$ is \textbf{perfect} if $E$ is closed and every point of $E$ is a
    limit point of $E$.

    \item $E$ is \textbf{bounded} if there exists a real number $M$ and a point
    $q \in X$ such that $d(p,q) < M$ for all $p \in E$.

    \item $E$ is \textbf{dense} in $X$ if every point of $X$ is a limit point of
    $E$, or a point of $E$ (or both).
\end{enumerate}





\end{frame}

\begin{frame}{Numerical Sequence and Series}

\textbf{Definition.}\footnote{Definition adapted from T. Tao, \emph{Analysis
I}.} Let $m$ be an integer.  
A sequence $(p_n)_{n=m}^{\infty}$ of rational numbers is any function
\[
f: \{ n \in \mathbb{Z} : n \geq m \} \;\to\; \mathbb{Q},
\]
i.e., a mapping which assigns to each integer $n \geq m$ a rational number
$p_n$.

\medskip
More informally, a sequence $(p_n)_{n=m}^{\infty}$ of rational numbers is a
collection:
\[
p_m, \; p_{m+1}, \; p_{m+2}, \; \dots
\]

\bigskip
\textbf{Example.}  
The sequence defined by $p_n = \tfrac{1}{n}$ for $n \geq 1$ is
\[
1, \; \tfrac{1}{2}, \; \tfrac{1}{3}, \; \tfrac{1}{4}, \; \dots
\]

    
\end{frame}

\begin{frame}{Convergent Sequences}
\textbf{Definition.}\footnote{Definition adapted from W. Rudin, \emph{Principles
of Mathematical Analysis}.} A sequence ${p_n}$ in a metric space $X$ is said to
\textbf{converge} if there is said a point $a \in X$ with the following
property: for every $\epsilon > 0$, there is an integer $N$ such that $d(p_n, p)
< \epsilon$ whenever $n > N$ implies $d(p_n,p)< \epsilon$.  In this case, we say
that the sequence converges to $p$, or that $p$ is the \textbf{limit} of the
sequence, and we write $\lim_{n \to \infty} p_n = p$ or $p_n \to p$ as $n \to
\infty$.\\ 
If ${p_n}$ does not converge, we say that the sequence \textbf{diverges}.\\ 

The set of points $p_n$ is called the \textbf{range} of the sequence. The
sequence is said to be \textbf{bounded}\footnote{Let $M \geq 0$ be rational.  A
finite sequence $p_1, p_2, \dotsc, p_n$ is bounded by $M$ iff $|p_i| \leq M$ for
all $1 \leq i \leq n$. An infinite sequence $\{a_n\}_{n=1}^{\infty}$ is bounded
by $M$ iff $|a_i| \leq M$ for all $i \geq 1$. } iff it is bounded by $M$ for
some rational $M \geq 0$. 
\end{frame}

\begin{frame}{Convergent Sequences}
\begin{theorem}
Let $(p_n)$ be a sequence in a metric space $X$.
\begin{enumerate}
    \item[(\textbf{a})] $(p_n)$ converges to $p \in X$ if and only if every
    neighborhood of $p$ contains $p_n$ for all but finitely many $n$.
    \item[(\textbf{b})] If $p \in X$, $p' \in X$, and if $(p_n)$ converges to
    $p$ and to $p'$, then $p' = p$.
    \item[(\textbf{c})] If $(p_n)$ converges, then $(p_n)$ is bounded.
    \item[(\textbf{d})] If $E \subset X$ and if $p$ is a limit point of $E$,
    then there is a sequence $(p_n)$ in $E$ such that $p = \lim_{n \to \infty}
    p_n$.
\end{enumerate}
\end{theorem}
\end{frame}
\begin{frame}{Convergent Sequences}
\begin{theorem}
Suppose $(s_n)$, $(t_n)$ are complex sequences and $\lim_{n \to \infty} s_n =
s$, $\lim_{n \to \infty} t_n = t$. Then
\begin{enumerate}
    \item[(\textbf{a})] $\lim_{n \to \infty} (s_n + t_n) = s + t$;
    \item[(\textbf{b})] $\lim_{n \to \infty} c s_n = c s$, $\lim_{n \to \infty}
    (c + s_n) = c + s$ for any number $c$;
    \item[(\textbf{c})] $\lim_{n \to \infty} s_n t_n = s t$;
    \item[(\textbf{d})] $\lim_{n \to \infty} (1/s_n) = 1/s$, provided $s_n \neq
    0$ ($n = 1, 2, 3, \dotsc$) and $s \neq 0$.
\end{enumerate}
\end{theorem}
\end{frame}

\begin{frame}{Convergent Sequences}
\begin{theorem}
(a) Suppose $\mathbf{x}_n \in \mathbb{R}^k$ ($n = 1, 2, 3, \dotsc$) and
$\mathbf{x}_n = (\alpha_{1,n}, \dotsc, \alpha_{k,n})$. Then $(\mathbf{x}_n)$
converges to $\mathbf{x} = (\alpha_1, \dotsc, \alpha_k)$ if and only if
\begin{equation}
 \lim_{n \to \infty} \alpha_{j,n} = \alpha_j \quad (1 \leq j \leq k).
\end{equation}



(b) Suppose $(\mathbf{x}_n)$, $(\mathbf{y}_n)$ are sequences in $\mathbb{R}^k$,
$(\beta_n)$ is a sequence of real numbers, and $\mathbf{x}_n \to \mathbf{x}$,
$\mathbf{y}_n \to \mathbf{y}$, $\beta_n \to \beta$. Then
\[
\lim_{n \to \infty} (\mathbf{x}_n + \mathbf{y}_n) = \mathbf{x} + \mathbf{y}, \quad 
\lim_{n \to \infty} \mathbf{x}_n \cdot \mathbf{y}_n = \mathbf{x} \cdot \mathbf{y}, \quad 
\lim_{n \to \infty} \beta_n \mathbf{x}_n = \beta \mathbf{x}.
\]
\end{theorem}
\end{frame}

\begin{frame}{Sub-Sequences}
\textbf{Definition.} Given a sequence $(p_n)$, consider a sequence $(n_k)$ of
positive integers such that $n_1 < n_2 < n_3 < \cdots$. Then the sequence
$(p_{n_k})$ is called a \textbf{subsequence} of $(p_n)$. If $(p_{n_k})$
converges, its limit is called a \textbf{subsequential limit} of $(p_n)$. \\ 

A sequence $(p_n)$ converges to $p$ if and only if every subsequence of $(p_n)$
converges to $p$.

\begin{theorem}
\begin{enumerate}
    \item[(\textbf{a})] If $(p_n)$ is a sequence in a compact metric space $X$,
    then some subsequence of $(p_n)$ converges to a point of $X$.
    \item[(\textbf{b})] Every bounded sequence in $\mathbb{R}^k$ contains a
    convergent subsequence.
\end{enumerate}
\end{theorem}

\begin{theorem}
The subsequential limits of a sequence $(p_n)$ in a metric space $X$ form a
closed subset of $X$.
\end{theorem}
\end{frame}



\begin{frame}{Cauchy Sequences}
\textbf{Definition.} A sequence $(p_n)$ in a metric space $X$ is called a
\textbf{Cauchy sequence} if for every $\epsilon > 0$ there is an integer $N$
such that $d(p_n, p_m) < \epsilon$ whenever $m, n > N$. \\ 

\textbf{Definition.} Let $E$ be a nonempty subset of a metric space $X$ and let
$S$ be the set of all real numbers of the form $d(p, q)$ with $p \in E$ and $q
\in E$. The $\sup S$ is called the \textbf{diameter} of $E$.\\ 

If $(p_n)$ is a sequence in $X$ and if $E_N$ consists of the points $p_N,
p_{N+1}, p_{N+2}, \dotsc$, it is clear from the two preceding definitions that
$(p_n)$ is a Cauchy sequence if and only if
\[
\lim_{N \to \infty} \text{diam } E_N = 0.
\]
\end{frame}
\begin{frame}{Cauchy Sequences}
  \begin{theorem}
\begin{enumerate}
    \item[(\textbf{a})] If $\overline{E}$ is the closure of a set $E$ in a
    metric space $X$, then
    \[
    \text{diam } \overline{E} = \text{diam } E.
    \]
    
    \item[(\textbf{b})] If $K_n$ is a sequence of compact sets in $X$ such that
    $K_n \supset K_{n+1}$ ($n = 1, 2, 3, \dotsc$) and if
    \[
    \lim_{n \to \infty} \text{diam } K_n = 0,
    \]
    then $\bigcap_{n=1}^{\infty} K_n$ consists of exactly one point.
\end{enumerate}
\end{theorem}
\end{frame} 
\begin{frame}{Cauchy Sequences}
  \begin{theorem}
\begin{enumerate}
    \item[(\textbf{a})] In any metric space $X$, every convergent sequence is a
    Cauchy sequence.
    \item[(\textbf{b})] If $X$ is a compact metric space and if $(p_n)$ is a
    Cauchy sequence in $X$, then $(p_n)$ converges to some point of $X$.
    \item[(\textbf{c})] In $\mathbb{R}^k$, every Cauchy sequence converges.
\end{enumerate}
\end{theorem}
\begin{block}{Note}
The difference between the definition of convergence and the definition of a
Cauchy sequence is that the limit is explicitly involved in the former, but not
in the latter.

The fact that a sequence converges in $\mathbb{R}^k$ if and only if it is a
Cauchy sequence is usually called the \textbf{Cauchy criterion for convergence}.
\end{block}
\end{frame}

\begin{frame}{Cauchy Sequences}
\textbf{Definition.} A metric space in which every Cauchy sequence converges is
said to be \textbf{complete}.\\

\textbf{Definition.} A sequence $(s_n)$ of real numbers is said to be
\begin{enumerate}
    \item[(\textbf{a})] \textbf{monotonically increasing} if $s_n \leq s_{n+1}$
    ($n = 1, 2, 3, \dotsc$);
    \item[(\textbf{b})] \textbf{monotonically decreasing} if $s_n \geq s_{n+1}$
    ($n = 1, 2, 3, \dotsc$).
\end{enumerate}

\begin{theorem}
Suppose $(s_n)$ is a monotonic sequence of real numbers. Then $(s_n)$ converges
if and only if it is bounded.
\end{theorem}
\end{frame}


\begin{frame}{Upper and Lower Limits}
\textbf{Definition.} Let $(s_n)$ be a sequence of real numbers with the
following property: For every real $M$ there is an integer $N$ such that $n \geq
N$ implies $s_n \geq M$. We then write
\[
s_n \to +\infty.
\]
Similarly, if for every real $M$ there exists an integer $N$ such that $n \geq
N$ implies $s_n \leq M$, we write
\[
s_n \to -\infty.
\]


\textbf{Definition.} Let $(s_n)$ be a sequence of real numbers. Let $E$ be the
set of numbers $x$ (in the extended real number system) such that $s_{n_k} \to
x$ for some subsequence $(s_{n_k})$. This set $E$ contains all subsequential
limits, plus possibly the numbers $+\infty$, $-\infty$.

\medskip % Adds a small vertical space
\noindent\textbf{Recall:}
\begin{itemize}
    \item The \textbf{supremum} ($\sup$), or \textbf{least upper bound}, of a
    set is the smallest real number (or $+\infty$) that is $\geq$ every element.
    \item The \textbf{infimum} ($\inf$), or \textbf{greatest lower bound}, of a
    set is the largest real number (or $-\infty$) that is $\leq$ every element.
\end{itemize}
\end{frame}
\begin{frame}{Upper and Lower Limits}
\medskip
We now put
\[
s^* = \sup E, \quad s_* = \inf E.
\]
The numbers $s^*$, $s_*$ are called the \textbf{upper} and \textbf{lower limits}
of $(s_n)$; we use the notation
\[
\limsup_{n \to \infty} s_n = s^*, \quad \liminf_{n \to \infty} s_n = s_*.
\]

\begin{theorem}
Let $(s_n)$ be a sequence of real numbers. Let $E$ and $s^*$ have the same
meaning as in Definition 3.16. Then $s^*$ has the following two properties:
\begin{enumerate}
    \item[(\textbf{a})] $s^* \in E$.
    \item[(\textbf{b})] If $x > s^*$, there is an integer $N$ such that $n \geq
    N$ implies $s_n < x$.
\end{enumerate}
Moreover, $s^*$ is the only number with the properties (a) and (b).
\end{theorem}

\end{frame}

\begin{frame}{Upper and Lower Limits} 

\begin{theorem}
If $s_n \leq t_n$ for $n \geq N$, where $N$ is fixed, then
\[
\liminf_{n \to \infty} s_n \leq \liminf_{n \to \infty} t_n, \quad
\limsup_{n \to \infty} s_n \leq \limsup_{n \to \infty} t_n.
\]
\end{theorem}
\end{frame}


\begin{frame}{Some Special Series}
  \begin{remark}
    If $0 \leq x_n \leq s_n$ for $n \geq N$, where $N$ is some fixed number, and
    if $s_n \to 0$, then $x_n \to 0$.
  \end{remark}
\end{frame}
\end{document}
