\documentclass{beamer}
\usepackage{amsmath}
\setbeamertemplate{theorems}[ams style]
% Theme
\usetheme{Madrid}
\usecolortheme{beaver}
\theoremstyle{plain}
\newtheorem{theorem}{Theorem}

% Title info
\title[Real Analysis]{Real Analysis}
\subtitle{Basic Topology}
\author[Abhijeet]{Abhijeet \\ \small abhijeet.m@igidr.ac.in}
\institute[IGIDR]{Mathematics for Economists \\  IGIDR}
\date{\today}

\begin{document}

% Title slide
\begin{frame}[plain]
  \titlepage
\end{frame}

% A simple second slide
\begin{frame}{Metric Spaces}
  \begin{itemize}

    \item A metric space is a set $X$ together with a function $d: X \times X
    \to \mathbb{R}$ satisfying:
    \begin{itemize}
      \item Non-negativity: $d(x,y) \geq 0$ and $d(x,y) = 0$ iff $x = y$.
      \item Symmetry: $d(x,y) = d(y,x)$.
      \item Triangle inequality: $d(x,z) \leq d(x,y) + d(y,z)$.
    \end{itemize}
    \item Example: The Euclidean space $\mathbb{R}^n$ with the standard distance
    $d(x,y) = \sqrt{\sum_{i=1}^n (x_i - y_i)^2}$.
    \item Metric spaces provide a framework for discussing concepts like
    convergence, continuity, and compactness.
  \end{itemize}
\end{frame}

\begin{frame}{Metric Spaces}
 Let $X$ be a metric space. All points and sets below are understood to be in
 $X$.


\begin{enumerate}[(a)]
    \item A \textbf{neighborhood} of $p$ is a set $N_r(p)$ consisting of all $q$
    such that $d(p,q) < r$ for some $r > 0$. The number $r$ is called the
    \emph{radius} of $N_r(p)$.

    \item A point $p$ is a \textbf{limit point} of the set $E$ if every
    neighborhood of $p$ contains a point $q \neq p$ such that $q \in E$.

    \item If $p \in E$ and $p$ is not a limit point of $E$, then $p$ is called
    an \textbf{isolated point} of $E$.

    \item $E$ is \textbf{closed} if every limit point of $E$ belongs to $E$.

    \item A point $p$ is an \textbf{interior point} of $E$ if there exists a
    neighborhood $N$ of $p$ such that $N \subseteq E$.

    \item $E$ is \textbf{open} if every point of $E$ is an interior point.
\end{enumerate}
\end{frame}
\begin{frame}{Metric Spaces}
\begin{enumerate}[(a)]
    \item The \textbf{complement} of $E$ (denoted $E^c$) is the set of all
    points $p \in X$ such that $p \notin E$.

    \item $E$ is \textbf{perfect} if $E$ is closed and every point of $E$ is a
    limit point of $E$.

    \item $E$ is \textbf{bounded} if there exists a real number $M$ and a point
    $q \in X$ such that $d(p,q) < M$ for all $p \in E$.

    \item $E$ is \textbf{dense} in $X$ if every point of $X$ is a limit point of
    $E$, or a point of $E$ (or both).
\end{enumerate}





\end{frame}

\begin{frame}{Numerical Sequence and Series}

\textbf{Definition.}\footnote{Definition adapted from T. Tao, \emph{Analysis
I}.} Let $m$ be an integer.  
A sequence $(a_n)_{n=m}^{\infty}$ of rational numbers is any function
\[
f: \{ n \in \mathbb{Z} : n \geq m \} \;\to\; \mathbb{Q},
\]
i.e., a mapping which assigns to each integer $n \geq m$ a rational number
$a_n$.

\medskip
More informally, a sequence $(a_n)_{n=m}^{\infty}$ of rational numbers is a
collection:
\[
a_m, \; a_{m+1}, \; a_{m+2}, \; \dots
\]

\bigskip
\textbf{Example.}  
The sequence defined by $a_n = \tfrac{1}{n}$ for $n \geq 1$ is
\[
1, \; \tfrac{1}{2}, \; \tfrac{1}{3}, \; \tfrac{1}{4}, \; \dots
\]

    
\end{frame}

\begin{frame}{Convergent Sequences}
\textbf{Definition.}\footnote{Definition adapted from W. Rudin, \emph{Principles
of Mathematical Analysis}.} A sequence ${a_n}$ in a metric space $X$ is said to
\textbf{converge} if there is said a point $a \in X$ with the following
property: for every $\epsilon > 0$, there is an integer $N$ such that $d(a_n, a)
< \epsilon$ whenever $n > N$ implies $d(a_n,a)< \epsilon$.  In this case, we say
that the sequence converges to $a$, or that $a$ is the \textbf{limit} of the
sequence, and we write $\lim_{n \to \infty} a_n = a$ or $a_n \to a$ as $n \to
\infty$.\\ 
If ${a_n}$ does not converge, we say that the sequence \textbf{diverges}.\\ 

The set of points $a_n$ is called the \textbf{range} of the sequence. The
sequence is said to be \textbf{bounded}\footnote{Let $M \geq 0$ be rational.  A
finite sequence $a_1, a_2, \dotsc, a_n$ is bounded by $M$ iff $|a_i| \leq M$ for
all $1 \leq i \leq n$. An infinite sequence $\{a_n\}_{n=1}^{\infty}$ is bounded
by $M$ iff $|a_i| \leq M$ for all $i \geq 1$. } iff it is bounded by $M$
for some rational $M \geq 0$. 
\end{frame}

\begin{theorem}
Let $(p_n)$ be a sequence in a metric space $X$.
\begin{enumerate}
    \item[(\textbf{a})] $(p_n)$ converges to $p \in X$ if and only if every neighborhood of $p$ contains $p_n$ for all but finitely many $n$.
    \item[(\textbf{b})] If $p \in X$, $p' \in X$, and if $(p_n)$ converges to $p$ and to $p'$, then $p' = p$.
    \item[(\textbf{c})] If $(p_n)$ converges, then $(p_n)$ is bounded.
    \item[(\textbf{d})] If $E \subset X$ and if $p$ is a limit point of $E$, then there is a sequence $(p_n)$ in $E$ such that $p = \lim_{n \to \infty} p_n$.
\end{enumerate}
\end{theorem}

\end{document}
