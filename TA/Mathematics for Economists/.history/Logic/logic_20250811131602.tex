\documentclass{beamer}
\usetheme{CambridgeUS}

\title{Logic}
\author{Abhijeet}
\institute{IGIDR}
\date{\today}

\begin{document}

\begin{frame}[plain]
  \titlepage
\end{frame}

\begin{frame}{Statements}
\begin{itemize}
    \item \textbf{Statements} are declarative or an assertion that is either true or false (but not both).
    \item (a) The integer 3 is odd. (b) The square root of 1 is 0. Both are statements. The former is true, the latter is false.
    \item For a sentence to be considered as a statement, it is not necessary that we know the truth value of the statement.
    \item The possible truth value of a \textbf{statement} can be listed in a truth table. The following is an example:
\end{itemize}

\begin{center}
\begin{tabular}{|c|c|c|c|c|c|c|c|c|}
    \hline
    $P$ & T & T & T & T & F & F & F & F \\
    \hline
    $Q$ & T & T & F & F & T & T & F & F \\
    \hline
    $R$ & T & F & T & F & T & F & T & F \\
    \hline
\end{tabular}
\end{center}

\end{frame}

\begin{frame}{Negations}
    \begin{itemize}
    \item The \textbf{negation} of a statment \textbf{P} is the statement \textbf{not P}.
    \item e.g., $P_{1}:$ The Sun rises in the east. $ \neg P_{1}:$ The Sun rises in the west. Statement $P_{1}$ is true while statment $ \neg P_{1}$ is false.
    \item The negation of a \textbf{true} statement is always false and vice versa. 
    \end{itemize}
\begin{center}
\begin{tabular}{|c|c|}
    \hline
    $P$ & $\neg P$ \\
    \hline
    T & F \\
    F & T \\
    \hline
\end{tabular}
\end{center}

\end{frame}

\begin{frame}{Disjunctions}
    \begin{itemize}
    \item The \textbf{disjunction} of the statement \textbf{P} and \textbf{Q} is the statement \textbf{P or Q} and is denoted by \textbf{P $\lor$ Q}.
    \item The disjunction statment is true if either one of P and Q is true, or if both P and Q is true. 
    \item For example, P: The integer 10 is even, Q: 5 is the cube root of 25. Here P is true and Q is false. The disjunction statement would be P$\lor$Q, which we would read, Either "The integer 10 is even" or "5 is a cube root of 25". And this disjunction statement is true, cause P is true. 
    \end{itemize}
    \begin{center}
\begin{tabular}{|c|c|c|c|c|}
    \hline
    $P$ & T & T & F & F \\
    \hline
    $Q$ & T & F & T & F \\
    \hline
    $P \lor Q$ & T & T & T & F \\
    \hline
\end{tabular}
\end{center}
\end{frame}

\begin{frame}{Conjunctions}
    \begin{itemize}
    \item The \textbf{conjunction} of the statement \textbf{P} and \textbf{Q} is the statement \textbf{P and Q} and is denoted by \textbf{P $\land$ Q}.
    \item The conjunction statment is true when both P and Q are true, otherwise the conjunction statement is false.
    \item For example, P: The integer 10 is even, Q: 5 is the cube root of 25. Here P is true and Q is false. The disjunction statement would be P$\land$Q, which we would read, "The integer 10 is even" and "5 is a cube root of 25". And this conjunction statement is False, cause Q is False.
    \end{itemize}
    \begin{center}
\begin{tabular}{|c|c|c|c|c|}
    \hline
    $P$ & T & T & F & F \\
    \hline
    $Q$ & T & F & T & F \\
    \hline
    $P \lor Q$ & T & F & F & F \\
    \hline
\end{tabular}
\end{center}
\end{frame}


\begin{frame}{Implications (Conditional)}
    \begin{itemize}
    \item For statements $P$ and $Q$, the implication (or conditional) is the statement
    $$\text{If } P, \text{ then } Q$$. in such a acse the statemnt is called "Vacuously true".
    and is denoted by $P \Rightarrow Q$. We can also say, "$P$ implies $Q$."
    \item $P \Rightarrow Q$ is false only when $P$ is true and $Q$ is false ($P \Rightarrow Q$ is true otherwise).
    \item For $P_1$: The integer 3 is odd. and $P_2$: The integer 57 is prime., the implication
    $P_1 \Rightarrow P_2 \text{: If 3 is an odd integer, then 57 is prime.}$
    is a false statement. The implication,$P_2 \Rightarrow P_1 \text{: If 57 is prime, then 3 is odd.}$
is true.
    \end{itemize}
    \begin{center}
\begin{tabular}{|c|c|c|c|c|}
    \hline
    $P$ & T & T & F & F \\
    \hline
    $Q$ & T & F & T & F \\
    \hline
    $P \Rightarrow Q$ & T & F & T & T \\
    \hline
\end{tabular}
\end{center}
\end{frame}

\begin{frame}{Implications (Conditional)}
    The following are equivalent: 
    \begin{itemize}
    
    \item If $P$, then $Q$.
    \item $Q$ if $P$.
    \item $P$ implies $Q$.
    \item $P$ only if $Q$.
    \item $P$ is sufficient for $Q$.
    \item $Q$ is necessary for $P$.
    \end{itemize}

\end{frame}

\end{document}