\documentclass{beamer}
\usetheme{CambridgeUS}

\title{Logic}
\author{Abhijeet}
\institute{IGIDR}
\date{\today}

\begin{document}

\begin{frame}[plain]
  \titlepage
\end{frame}

\begin{frame}

\end{frame}{Statements}
\begin{itemize}
    \item textbf{Statements} are declarative or an assertion that is either true or false (but not both). 
    \item (a) The integer 3 is odd. (b) The square root of 1 is 0. Both are statements. The former is true, the latter is false.
    \item For a sentence to be considered as a statement, it is not necesssary that we know the truth value of the statment. 
    \item The possible truth value of a \textbf{statement} can be listed in a truth table. The following is an example: 
    
\end{itemize}
\begin{tabular}{|c|c|c|}
    \hline
    $P$ & $Q$ & $R$ \\
    \hline
    T & T & T \\
    T & T & F \\
    T & F & T \\
    T & F & F \\
    F & T & T \\
    F & T & F \\
    F & F & T \\
    F & F & F \\
    \hline
\end{tabular}
\end{frame}

\end{document}