\documentclass{beamer}
\usetheme{CambridgeUS}

\title{Logic}
\author{Abhijeet}
\institute{IGIDR}
\date{\today}

\begin{document}

\begin{frame}[plain]
  \titlepage
\end{frame}

\begin{frame}{Statements}
\begin{itemize}
    \item \textbf{Statements} are declarative or an assertion that is either true or false (but not both).
    \item (a) The integer 3 is odd. (b) The square root of 1 is 0. Both are statements. The former is true, the latter is false.
    \item For a sentence to be considered as a statement, it is not necessary that we know the truth value of the statement.
    \item The possible truth value of a \textbf{statement} can be listed in a truth table. The following is an example:
\end{itemize}

\begin{center}
\begin{tabular}{|c|c|c|c|c|c|c|c|c|}
    \hline
    $P$ & T & T & T & T & F & F & F & F \\
    \hline
    $Q$ & T & T & F & F & T & T & F & F \\
    \hline
    $R$ & T & F & T & F & T & F & T & F \\
    \hline
\end{tabular}
\end{center}

\end{frame}

\begin{frame}{Negations}
    \begin{itemize}
    \item The \textbf{negation} of a statment \textbf{P} is the statement \textbf{not P}.
    \item e.g., $P_{1}:$ The Sun rises in the east. $ \neg P_{1}:$ The Sun rises in the west. Statement $P_{1}$ is true while statment $ \neg P_{1}$ is false.
    \item The negation of a \textbf{true} statement is always false and vice versa. 
    \end{itemize}
\begin{center}
\begin{tabular}{|c|c|}
    \hline
    $P$ & $\neg P$ \\
    \hline
    T & F \\
    F & T \\
    \hline
\end{tabular}
\end{center}

\end{frame}

\begin{frame}{Disjunctions}
    \begin{itemize}
    \item The \textbf{disjunction} of the statement \textbf{P} and \textbf{Q} is the statement \textbf{P or Q} and is denoted by \textbf{P $\lor$ Q}.
    \item The disjunction statment is true if either one of P and Q is true, or if both P and Q is true. 
    \item For example, P: The integer 10 is even, Q: 5 is the cube root of 25. Here P is true and Q is false. The disjunction statement would be P$\lor$Q, which we would read, Either "The integer 10 is even" or "5 is a cube root of 25". And this disjunction statement is true, cause P is true. 
    \end{itemize}
    \begin{center}
\begin{tabular}{|c|c|c|c|c|}
    \hline
    $P$ & T & T & F & F \\
    \hline
    $Q$ & T & F & T & F \\
    \hline
    $P \lor Q$ & T & T & T & F \\
    \hline
\end{tabular}
\end{center}
\end{frame}

\begin{frame}{Conjunctions}
    \begin{itemize}
    \item The \textbf{conjunction} of the statement \textbf{P} and \textbf{Q} is the statement \textbf{P and Q} and is denoted by \textbf{P $\land$ Q}.
    \item The conjunction statment is true when both P and Q are true, otherwise the conjunction statement is false.
    \item For example, P: The integer 10 is even, Q: 5 is the cube root of 25. Here P is true and Q is false. The disjunction statement would be P$\land$Q, which we would read, "The integer 10 is even" and "5 is a cube root of 25". And this conjunction statement is False, cause Q is False.
    \end{itemize}
    \begin{center}
\begin{tabular}{|c|c|c|c|c|}
    \hline
    $P$ & T & T & F & F \\
    \hline
    $Q$ & T & F & T & F \\
    \hline
    $P \lor Q$ & T & F & F & F \\
    \hline
\end{tabular}
\end{center}
\end{frame}


\begin{frame}{Implications (Conditional)}
    \begin{itemize}
    \item For statements $P$ and $Q$, the implication (or conditional) is the statement
    $$\text{If } P, \text{ then } Q$$. in such a acse the statemnt is called "Vacuously true".
    and is denoted by $P \Rightarrow Q$. We can also say, "$P$ implies $Q$."
    \item $P \Rightarrow Q$ is false only when $P$ is true and $Q$ is false ($P \Rightarrow Q$ is true otherwise).
    \item For $P_1$: The integer 3 is odd. and $P_2$: The integer 57 is prime., the implication
    $P_1 \Rightarrow P_2 \text{: If 3 is an odd integer, then 57 is prime.}$
    is a false statement. The implication,$P_2 \Rightarrow P_1 \text{: If 57 is prime, then 3 is odd.}$
is true.
    \end{itemize}
    \begin{center}
\begin{tabular}{|c|c|c|c|c|}
    \hline
    $P$ & T & T & F & F \\
    \hline
    $Q$ & T & F & T & F \\
    \hline
    $P \Rightarrow Q$ & T & F & T & T \\
    \hline
\end{tabular}
\end{center}
\end{frame}

\begin{frame}{Implications (Conditional)}
    The statement or open sentence $P$ in the implication $P \Rightarrow Q$ is commonly referred to as the \textbf{hypothesis} or \textbf{premise} of $P \Rightarrow Q$, while $Q$ is called the \textbf{conclusion} of $P \Rightarrow Q$.\\

    \vspace{0.2cm}
    The following are equivalent: 
    \begin{itemize}
    
    \item If $P$, then $Q$.
    \item $Q$ if $P$.
    \item $P$ implies $Q$.
    \item $P$ only if $Q$.
    \item $P$ is sufficient for $Q$.
    \item $Q$ is necessary for $P$.
    \end{itemize}

Use, P: You get A+ and Q: I give you a favour.
\end{frame}

\begin{frame}{Bi-Conditional}
\begin{itemize}
    \item For statements (or open sentences) $P$ and $Q$, the conjunction $(P \Rightarrow Q) \land (Q \Rightarrow P)$ of the implication $P \Rightarrow Q$ and its converse is called the biconditional of $P$ and $Q$ and is denoted by $P \Leftrightarrow Q$.
\end{itemize}

\begin{center}
\begin{tabular}{|c|c|c|c|c|}
    \hline
    $P$ & T & T & F & F \\
    \hline
    $Q$ & T & F & T & F \\
    \hline
    $P \Leftrightarrow Q$ & T & F & F & T \\
    \hline
\end{tabular}
\end{center}

\begin{itemize}
    \item The following are equivalent ways to say $P \Leftrightarrow Q$:
    \begin{itemize}
        \item $P$ is equivalent to $Q$
        \item $P$ if and only if $Q$
        \item $P$ is necessary and sufficient for $Q$
    \end{itemize}
\end{itemize}
\end{frame}

\begin{frame}{Examples: 1}
    State the negations of the following statements: 
\begin{enumerate}
    \item 2 is a rational number.
    \item Two sides of the triangle have the same length.
    \item The area of the circle is at least $9\pi$.
    \item The point $P$ in the plane lies outside of the circle $C$.
\end{enumerate}
Complete the truth table
% \paragraph{Negations}
% \begin{enumerate}
%     \item 2 is an irrational number.
%     \item No two sides of the triangle have the same length (or all three sides have different lengths).
%     \item The area of the circle is less than $9\pi$.
%     \item The point $P$ in the plane lies inside or on the circle $C$.
% \end{enumerate}
\begin{center}
\begin{tabular}{|c|c|c|c|}
    \hline
    $P$ & $Q$ & $\neg P$ & $\neg Q$ \\
    \hline
    T & T & & \\
    T & F & & \\
    F & T & & \\
    F & F & & \\
    \hline
\end{tabular}
\end{center}
    
\end{frame}


\begin{frame}{Examples: 2}
Let $P$: 15 is odd. and $Q$: 21 is prime.
State each of the following in words and determine whether it is true or false.
\begin{enumerate}[(a)]
    \item $P \lor Q$
    \item $P \land Q$
    \item $(\neg P) \lor Q$
    \item $P \land (\neg Q)$
\end{enumerate}
Complete the truth table: \begin{center}
\begin{tabular}{|c|c|c|c|}
    \hline
    $P$ & $Q$ & $\neg Q$ & $P \land (\neg Q)$ \\
    \hline
    T & T &  &  \\
    T & F &  &  \\
    F & T &  &  \\
    F & F &  &  \\
    \hline
\end{tabular}
\end{center}
    
\end{frame}


\begin{frame}{Examples: 3}
Consider the statements $P$: 2 is rational. and $Q$: $22/7$ is rational.
Write each of the following statements in words and indicate whether it is true or false.
\begin{enumerate}[(a)]
    \item $P \Rightarrow Q$
    \item $Q \Rightarrow P$
    \item $(\neg P) \Rightarrow (\neg Q)$
    \item $(\neg Q) \Rightarrow (\neg P)$
\end{enumerate}
\end{frame}

\begin{frame}{Examples: 4}
A college student makes the following statement:
\begin{center}
    If I receive an A in both Calculus I and Discrete Mathematics this semester, then I'll take either Calculus II or Computer Programming this summer.
\end{center}
For each of the following, determine whether the statement above is true or false.
\begin{enumerate}[(a)]
    \item The student doesn't get an A in Calculus I but decides to take Calculus II this summer anyway.
    \item The student gets an A in both Calculus I and Discrete Mathematics but decides not to take any class this summer.
    \item The student does not get an A in Calculus I and decides not to take Calculus II but takes Computer Programming this summer.
    \item The student gets an A in both Calculus I and Discrete Mathematics and decides to take both Calculus II and Computer Programming this summer.

\end{enumerate}
\end{frame}

\begin{frame}{Examples: 5}
Let $P(x)$: $x$ is odd. and $Q(x)$: $x^2$ is odd. be open sentences over the domain $\mathbb{Z}$. State $P(x) \Leftrightarrow Q(x)$ in two ways: (1) using ``if and only if'' and (2) using ``necessary and sufficient.''\\
\vspace{0.4cm}
Consider the open sentences:
$P(x) : x = -2$. and $Q(x) : x^2 = 4$.
over the domain $S = \{-2, 0, 2\}$. State each of the following in words and determine all values of $x \in S$ for which the resulting statements are true.
\begin{enumerate}[(a)]
    \item $\neg P(x)$
    \item $P(x) \lor Q(x)$
    \item $P(x) \land Q(x)$
    \item $Q(x) \Rightarrow P(x)$
    \item $P(x) \Leftrightarrow Q(x)$
\end{enumerate}

\end{frame}


\begin{frame}{Tautologies and Contradictions}
    \begin{enumerate}
        \item The symbols $\neg$, $\lor$, $\land$, $\Rightarrow$ and $\Leftrightarrow$ are referred to as logical connectives. A compound statement is a statement composed of one or more given statements and at least one logical connective. For example, for a given component statement $P$, its negation $\neg P$ is a compound statement.
        \item A compound statement $S$ is called a \textbf{tautology} if it is true for all possible combinations of truth values of the component statements that comprise $S$. Hence, $P \lor (\neg P)$ is a tautology.
        \item A compound statement $S$ is called a \textbf{contradiction} if it is false for all possible combinations of truth values of the component statements that are used to form $S$. The statement $P \land (\neg P)$ is a contradiction. If a compound statement $S$ is a tautology, then its negation $\neg S$ is a contradiction.
    \end{enumerate}
\end{frame}

\begin{frame}{Examples 6}
\begin{itemize}
    \item For statements $P$ and $Q$, show that $P \Rightarrow (P \lor Q)$ is a tautology.
    \item For statements $P$ and $Q$, show that $(P \land (\neg Q)) \land (P \land Q)$ and $(P \Rightarrow \neg Q) \land (P \land Q)$ are contradictions.
    \item For statements $P$ and $Q$, show that $(P \land (P \Rightarrow Q)) \Rightarrow Q$ is a tautology. 
    \item For statements $P$, $Q$ and $R$, show that $((P \Rightarrow Q) \land (Q \Rightarrow R)) \Rightarrow (P \Rightarrow R)$ is a tautology.
    \item Let $R$ and $S$ be compound statements involving the same component statements. If $R$ is a tautology and $S$ is a contradiction, then what can be said of the following?
\end{itemize}
\end{frame}

\begin{frame}{Logical equivalence}
Let $R$ and $S$ be two compound statements involving the same component statements. Then $R$ and $S$ are called \textbf{logically equivalent} if $R$ and $S$ have the same truth values for all combinations of truth values of their component statements.\\
    
    \begin{center}
\begin{tabular}{|c|c|c|c|c|}
    \hline
    $P$ & $Q$ & $\neg P$ & $P \Rightarrow Q$ & $(\neg P) \lor Q$ \\
    \hline
    T & T & F & T & T \\
    T & F & F & F & F \\
    F & T & T & T & T \\
    F & F & T & T & T \\
    \hline
\end{tabular}
\end{center}

P: You earn an A on the final exam, Q: you will receive a Chocolate. How would we know that the instructor didn't keep his promise? Only when "P" is true and "Q" is false (In the case of P $\Rightarrow$ Q). But how do we express in a single sentence that the teacher kept the promise? Either "you did not get an A" OR "You received a chocolate". That is, $\neg P \lor Q$.

\end{frame}

\begin{frame}{Examples 7}
\begin{itemize}
    \item Let $P$ and $Q$ be statements.
    \begin{enumerate}[(a)]
        \item Is $\sim (P \lor Q)$ logically equivalent to $(\sim P) \lor (\sim Q)$? Explain.
        \item What can you say about the biconditional $\sim (P \lor Q) \Leftrightarrow ((\sim P) \lor (\sim Q))$?
    \end{enumerate}
    \item For statements $P$, $Q$ and $R$, use a truth table to show that each of the following pairs of statements are logically equivalent.
    \begin{enumerate}[(a)]
        \item $(P \land Q) \Leftrightarrow P$ and $P \Rightarrow Q$.
        \item $P \Rightarrow (Q \lor R)$ and $(\sim Q) \Rightarrow ((\sim P) \lor R)$.
    \end{enumerate}
    \item For statements $P$ and $Q$, show that $(\sim Q) \Rightarrow (P \land (\sim P))$ and $Q$ are logically equivalent.
    \item For statements $P$, $Q$ and $R$, show that $(P \lor Q) \Rightarrow R$ and $(P \Rightarrow R) \land (Q \Rightarrow R)$ are logically equivalent.
\end{itemize}
    
\end{frame}

\begin{frame}{Some fundamental Properties of Logical Equivalence}
For statements $P$, $Q$ and $R$, Thefollowing are logically equivalent.
\begin{itemize}
    \item \textbf{Commutative Laws}
    \begin{enumerate}[(a)]
        \item $P \lor Q \equiv Q \lor P$.
        \item $P \land Q \equiv Q \land P$.
    \end{enumerate}
    \item \textbf{Associative Laws}
    \begin{enumerate}[(a)]
        \item $P \lor (Q \lor R) \equiv (P \lor Q) \lor R$.
        \item $P \land (Q \land R) \equiv (P \land Q) \land R$.
    \end{enumerate}
    \item \textbf{Distributive Laws}
    \begin{enumerate}[(a)]
        \item $P \lor (Q \land R) \equiv (P \lor Q) \land (P \lor R)$.
        \item $P \land (Q \lor R) \equiv (P \land Q) \lor (P \land R)$.
    \end{enumerate}
    \item \textbf{De Morgan’s Laws}
    \begin{enumerate}[(a)]
        \item $\sim (P \lor Q) \equiv (\sim P) \land (\sim Q)$.
        \item $\sim (P \land Q) \equiv (\sim P) \lor (\sim Q)$.
    \end{enumerate}
\end{itemize}
\end{frame}

\begin{frame}{Examples 8}
\begin{itemize}
    \item Let $P$, $Q$ and $R$ be statements. Then $P \lor (Q \land R)$ and $(P \lor Q) \land (P \lor R)$ are logically equivalent.
    \item Let $P$ and $Q$ be statements. Then $\sim (P \lor Q)$ and $(\sim P) \land (\sim Q)$ are logically equivalent.
    \item For a real number $x$, let $P(x): x^2 = 2$. and $Q(x): x = 2$. State the negation of the biconditional $P \Leftrightarrow Q$ in words.
    \item Let $P$ and $Q$ be statements. Show that $[(P \lor Q) \land \sim (P \land Q)] \equiv \sim (P \Leftrightarrow Q)$.
\end{itemize}
\end{frame}

\begin{frame}{Quantified Statements- For All}
\begin{itemize}
    \item If $P(x)$ is an open sentence over a domain $S$, then $P(x)$ is a statement for each $x \in S$.
    \item An open sentence can be converted into a statement, namely by a method called quantification.
    \item Let $P(x)$ be an open sentence over a domain $S$. Adding the phrase ``For every $x \in S$'' to $P(x)$ produces a statement called a quantified statement.
    \item The phrase ``for every'' is referred to as the universal quantifier and is denoted by the symbol $\forall$. Other ways to express the universal quantifier are ``for each'' and ``for all.''
\end{itemize}
\end{frame}

\begin{frame}{Quantified Statements- There Exists}
\begin{itemize}
    \item Each of the phrases ``there exists,'' ``there is,'' ``for some,'' and ``for at least one'' is referred to as an \textbf{existential quantifier} and is denoted by the symbol $\exists$. The quantified statement
$$ \exists x \in S, P(x) $$
can be expressed in words by
``There exists $x \in S$ such that $P(x)$.''
    \item An open sentence can be converted into a statement, namely by a method called quantification.
 \item The existential quantifier is used to claim that at least one statement resulting from a given open sentence is true when the values of a variable are assigned from its domain. We know that for an open sentence $P(x)$ over a domain $S$, the quantified statement $\exists x \in S, P(x)$ is true provided $P(x)$ is a true statement for at least one element $x \in S$.
 
\end{itemize}
\end{frame}

\begin{frame}{Quantified statements}
    \begin{itemize}

    \item The statement $\exists x \in \mathbb{R}, x^2 > 0$ is true since, for example, $x^2 > 0$ is true for $x = 1$.
    \item The quantified statement $\exists x \in \mathbb{R}, 3x = 12$ is therefore true since there is some real number $x$ for which $3x = 12$, namely $x = 4$ has this property.
\end{itemize}
\end{frame}

\begin{frame}{Examples 9}
Consider the statement
$$ \text{For every two real numbers } x \text{ and } y, x^2 + y^2 \ge 0. $$
If we let $P(x, y) : x^2 + y^2 \ge 0$ where the domain of both $x$ and $y$ is $\mathbb{R}$, then this statement can be expressed as
$$ \forall x \in \mathbb{R}, \forall y \in \mathbb{R}, P(x, y) $$
or as
$$ \forall y \in \mathbb{R}, \forall x \in \mathbb{R}, P(x, y), \quad \text{or as} \quad \forall x, y \in \mathbb{R}, P(x, y) \quad \text{or} \quad \forall y, x \in \mathbb{R}, P(x, y) $$
since quantifiers of the same type commute. Since $x^2 \ge 0$ and $y^2 \ge 0$ for all real numbers $x$ and $y$, it follows that $x^2 + y^2 \ge 0$ and so $P(x, y)$ is true for all real numbers $x$ and $y$. Thus, this quantified statement is true.
\end{frame}

\begin{frame}{Examples 9}
The negation of this statement is therefore
\begin{align*}
    \sim (\forall x \in \mathbb{R}, \forall y \in \mathbb{R}, P(x, y)) &\equiv \exists x \in \mathbb{R}, \exists y \in \mathbb{R}, \sim P(x, y) \\
    &\equiv \exists x, y \in \mathbb{R}, \sim P(x, y)
\end{align*}
which, in words, is
$$ \text{There exist real numbers } x \text{ and } y \text{ such that } x^2 + y^2 < 0. $$
The resulting statement is therefore false.

For an open sentence containing two variables, the domains of the variables need not be the same.
\vspace{0.2cm}
State the negations of the following quantified statements:
\begin{enumerate}[(a)]
    \item For every rational number $r$, the number $1/r$ is rational.
    \item There exists a rational number $r$ such that $r^2 = 2$.
\end{enumerate}
\end{frame}

\end{document}